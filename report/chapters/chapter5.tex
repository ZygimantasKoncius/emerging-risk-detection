\chapter{Conclusion}

\section{Results}
The end result of this project is a system that parses financial reports, provides analytical insights by using term frequency and TF-IDF methods, then filters out the sentences relating to constant risks (TF-IDF classification), clusters those sentences and then using LDA on differently organised remaining data extracts emerging new topics. The whole system was built in a modular structure and every part was separately evaluated. The final output is not perfect and there is a lot of room for improvement. However, with appropriate financial market knowledge it can still be very helpful in identifying the emerging risks. The methods explored in this project and their results will hopefully provide community a little bit more data about the efficiency of various NLP techniques in the context of financial documents.

\section{Reflection}
During the course of this project I built and designed a whole NLP system. From the beginning, constant planning, time management and experimentation skills were crucial and improved very quickly. The first step was data preprocessing, which proved to be more challenging and took longer than expected, because of the document inconsistencies as well as the size of the dataset. This taught me how to work with big dataset as well as how to readjust timelines during the project. After that, I had to explore the dataset and gain better understanding about financial documents by acquiring some knowledge about financial markets. During the classifier design and development I learnt a lot about classification techniques, TF-IDF, cosine similarity and how to tweak classifier parameters based on the outcomes, which was particularly challenging. The next step was grouping sentences by similarity, for which I had to learn about clustering and word embeddings, which were very broad and complex topics. The main challenge when implementing clustering was finding and understanding less common clustering techniques that did not require a defined number of topics and could use cosine similarity kernel. The final step was topic modelling using LDA, which required a lot of experimentation with data representation and LDA parameters. Throughout, this project I had a lot of opportunities to design the system and its components, experiment, explore and implement functionalities using NLP and machine learning packages in Python.

\section{Future Improvements and Research}
There is a lot of potential for future research in this topic. First of all, as this project was built without a golden standard or any other baselines, most of its parts were difficult to test and evaluate, therefore it would be useful to generate a variety of curated data for this dataset. 

Furthermore, the project was built in a modular fashion to allow for independent improvements. One of the core areas that could be improved is classification, which is currently implemented by using TF-IDF. While it does provide basic classification, it most likely could be hugely improved by using more sophisticated methods such as "Seed-Guided Topic Model for Document Filtering and Classification"\cite{seedDFCpaper}. Alternative clustering solutions and their outcomes could also be further explored, with suggestion to specifically explore spherical K-Means. Same applies to topic modelling solutions in which further LDA exploration and tweaking should be considered.

Finally, some additional features that were out of scope for this project could be implemented. For example, detecting the first time the risks are mentioned and the companies that mentioned them as well as the emerging risk trends over time. For such feature implementation classification and clustering parts could probably be reused. Another, potentially very important feature could be sentiment analysis implementation. This project was done assuming that most sentences in risk sections represent some risk, however in reality a sizeable portion of sentences are in fact concerned with mitigating those risks and should be treated differently.

Despite of all of the mentioned future opportunities and improvements, the main objectives were accomplished and the results were satisfactory, therefore the project was a success.